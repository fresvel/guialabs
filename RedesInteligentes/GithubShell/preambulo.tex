\usepackage{enumitem}
\usepackage{xcolor}
\usepackage{listings}

\usepackage{fancyhdr}
\usepackage{graphicx}


\usepackage[most]{tcolorbox}
\usepackage{graphicx}
\definecolor{darkcol}{RGB}{ 7, 7, 50 }
\definecolor{linkcol}{RGB}{ 7, 70, 150 }
\definecolor{cmdcol}{RGB}{ 7, 100, 150 }
\usepackage{tabularx}
\usepackage{ifthen}
\usepackage{xstring, xifthen}

\usepackage{float}
\usepackage{hyperref}

\usepackage{geometry}
\geometry{
 a4paper,
 total={155mm,247mm},
 left=30mm,
 top=0mm,
 %top=25mm
 }

% Configuración de estilo para comandos en consola



\newtcolorbox{terbox}{
    colback=cmdcol!10,
    colupper=darkcol,
    colframe=cmdcol,
    %sharp corners,
    boxrule=1pt,
    left=10pt,
    right=10pt,
    top=5pt,
    bottom=5pt,
    breakable,
    fontupper=\ttfamily\fontsize{12}{14}\selectfont,
}

\hypersetup{
    colorlinks=true, % Desactiva el color de los enlaces
    %pdfborder={0 0 0} % Elimina el borde de los enlaces
    urlcolor=linkcol!70
}





\usepackage{caption}

\captionsetup{
    font=small, % Tamaño de la fuente (puedes usar tiny, small, normalsize, large, etc.)
    labelfont={bf, it}, % Estilo de la etiqueta (puedes usar bf para negrita, it para cursiva, etc.)
    labelsep=colon, % Separador entre la etiqueta y el texto del caption (puedes usar colon, period, space, etc.)
    textfont=it, % Estilo del texto del caption
    justification=centering, % Alineación del texto (centering, raggedright, raggedleft)
    singlelinecheck=false, % Alineación para captions de una sola línea
}



\renewcommand{\figurename}{Ilustración} % Cambia "Figura" a "Ilustración"





\setlength{\parindent}{0pt}










\input{header} %geometry top 0